\section{Webservice mit REST}
\label{GrundlagenREST}
\ac{REST} ( Representation State Trasfer) wird vor allem wegen seiner Wiederverwendbarkeit, Skalierbarkeit, Erweiterbarkeit und eine einfache Integration in Web Services geschätzt. Das zentrale Konzept von REST sind Ressourcen, weshalb in diesem Umfeld auch von einer Ressource Oriented Architecture (\ac{ROA}) gesprochen wird.


\ac{REST} wurde erstmals im Jahr 2000 in der Dissertation Architectural Styles and the Design of Network-based Soft-ware Architectures von Roy Fielding vorgestellt. Er hatte das im Jahr 1994 entworfene \ac{HTTP} Objekt Model weiterentwickelt.
Nach \cite{tilkov2011rest} lässt sich \ac{REST} auf 5 Kernprinzipien zusammenfassen:

\begin{itemize}
	\item Eindeutig identifizierbare Ressourcen
	\item Verknüpfung von Ressourcen mithilfe von Links/Hypermedia
	\item Standardmethoden
	\item Repräsentation der Ressourcen in verschiedenen Formaten
	\item Zustandslose Kommunikation
\end{itemize}

Nach dem \ac{REST} Prinzip sind Ressourcen eindeutig über Uniform Ressource Identifier (\ac{URI}) erreichbar. So können Ressourcen ohne Hintergrundinformationen aufgerufen werden. Ein weiterer Begriff in diesem Umfeld ist Hypermedia, das für Inhalte steht die miteinander verknüpft sind und in dessen sich ein Programm/ Nutzer frei bewegen kann. 
Die Standartmethoden geben an was mit der Ressource gemacht werden soll. Jede Ressource sollte folgende Methoden GET, POST, PUT, DELETE, HEAD und OPTIONS implementiert haben. Bei GET handelt es sich um eine sichere Methode, da nur eine Ressource beim Server angefordert wird und so kein Nebeneffekt entstehen kann. PUT verändert bereits bestehende Daten, wobei DELETE vorhandene Ressourcen löscht. Bei PUT und DELETE sowie GET handelt es sich um indempotente Methoden, da bei einem erneuten Aufruf der Methoden keine Nebeneffekte auftreten. Nur bei der POST-Methode gibt es bei erneutem Aufrufen der Methode keine Garantie, dass Nebeneffekte auftreten.
\paragraph{}
Die zustandslose Kommunikation zwischen Client und Server ist ein weiterer wichtiger Punkt von \ac{REST}. Es werden bei einem Aufruf einer Ressource keinerlei Informationen zwischen dem Aufruf mehreren Ressourcen zwischengespeichert. Der Vorteil von zustandsloser Kommunikation ist, dass ein Webservice leicht Skalierbar ist und so etwa bei großen Services sehr einfach Anfragen an mehrere Server verteilt werden können. Dieses Prinzip wird Lastenverteilung genannt.
\paragraph{}
Ein weiterer Performancevorteil bietet das Caching. Da jeder GET eine eindeutige Adresse hat kann festgestellt werden, ob eine Ressource bereits angefordert wurde und wenn ja, die  Ressource aus dem Cache verwenden. Durch dieses Prinzip können Anfragen beim Server reduziert werden.
\paragraph{}
Zum Thema der Sicherheit sagt der Erfinder von REST Roy T. Fielding: "`RESTful systems perform secure operations in the same way as any messaging protocol: either by encapsulating the message stream (SSL, TLS, SSH, IPsec) or by encrypting the messages (PGP, S/MIME, etc.). There are numerous Examples of that in practice, and more in the future once browsers learn how to implement other authentication mechanisms."' \cite{fielding2008restapi}. Es lässt sich jede Sicherheitstechnologie in ein REST Service einbauen, wie zum Beispiel  \ac{SSL}, \ac{HTTPS} oder auch OAuth. Dies ermöglicht einen Einsatz von REST-Anwendungen im Unternehmensumfeld.

Bei Hypermedia as the Engine of Application State(\ac{HATEOAS}) handelt es sich um ein Entwurfsmuster. Das Ziel ist es, einer Ressource einen Link mitzugeben der zu einer Folgeoperation führt. Durch diese lose Kopplung kann die Schnittstelle angepasst werden ohne am Client festgelegte Einstellungen zu ändern.
\paragraph{}
Zusammenfassend lässt sich über \ac{REST} sagen, dass es sich um ein Ressourcenorientiertes Design handelt das durch die Zustandslosigkeit einer Operation viele Vorteile im Bereich Load-Balancing und Chaching  bietet. Durch die einfache Implementation der REST-Schnittstelle sind Services leicht zu warten, erweiterbar und skalierbar.
