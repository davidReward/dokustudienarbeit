\section{Ausblick}
-Registrierung neuer Sensorknoten mittels Webapp
-Routing
-richtiges Mesh
-verschlüsellung
-Datamining
-Aktorknoten

-Auflösung verringern von Daten

Bei der Realisierung dieses Projekts kamen Ideen für die Fortsetzung der Arbeit auf. So wäre eine benutzerfreundliche Möglichkeit weitere Sensorknoten zu registrieren mit Sicherheit ein gefragtes Feature. Man könnte dies mit eines Webapplikation umsetzten.
Der Vorhanden "Mesh-Algorithmus" ist in Zukünftigen Arbeiten ersetzbar durch einen Mesh-Algorithmus wie er in der Literatur definiert ist. Dazu kann man beispielsweise einen entsprechenden RFC implementieren. In diesem Zuge ist es auch möglich eine Routing-Metrik für das neu implementierte Protokoll zu designen und die Datenpakete zu verschlüsseln. 

Die gesammelten Sensordaten sind auch als Basis für Dataminig nutzbar

Ferner kann auch der Gegenspieler zum Sensorknoten realisiert werden, ein Aktorknoten.

 