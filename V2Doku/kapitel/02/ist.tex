\section{Ist-Zustand}
\label{sec:IstZustand}
Für die Durchführung der Studienarbeit sind bereits zwei Hardwarekomponenten gegeben. Weshalb zunächst auf beide eigesetzten Systeme kurz eingegangen wird.
\paragraph{Raspberry Pi} 
Bei einem Raspberry Pi handelt es sich um einen kleinen Einplatinencomputer. Dieser kann vor allem für kleinere Hard-/Software Projekte eingesetzt werden. Der Raspberry Pi überzeugt insbesondere durch seine Kompaktheit und seine Erweiterungsfähigkeit. An den \ac{GPIO} Schnittstellen können zusätzliche Module für Sensoren, LEDs und Funkmodule angelschlossen werden.  Durch ein integriertes WLAN Modul und einem LAN Modul bietet der Raspberry Pi alle benötigten Komponenten um als Gateway für das Internet der Dinge zu dienen. Der Raspberry Pi kann mit vielen verschiedenen Programmiersprachen programmiert werden, wie zum Beipiel Java, C , C++ oder auch Python. In Kapitel \ref{sec:raspiBasics} wird genauer auf den Aufbau des Raspberry Pi’s eingegangen.  
\paragraph{Arduino} 
Bei einem Arduino handelt es sich um ein Board mit einem Mikrocontroller. Der Arduino basiert auf einem Atmel-AVR-Mikrocontroller der megaAVR-Serie. Dieser wird mit 5V betrieben und bietet mehrere digitale und analoge Input- / Output-Pins.  Diese Pins können genutzt werden um verschiedene Sensoren, Aktoren und Funkmodule anzuschließen. Der Arduino kann mit Hilfe des eigens entwickeltem Arduino-IDE per USB Schnittstelle programmiert werden. Die Programmierung erfolgt in C bzw. C++. Der Arduino zeichnet sich vor allem durch seine Kompaktheit und seinem niedrigem Energieverbrauch aus. Für Projekte bezüglich des Internets der Dinge sind dies optimale  Eigenschaften.   In Kapitel \ref{sec:Arduino} wird genauer auf die Eigenschaften und Aufbau des Arduinos eingegangen.

Für die Fertigstellung des Projekts stehen die beiden Theoriesemester des 5. und 6. Semesters des Studiums zum Bachelor of Engineering Informationstechnik bereit. Außer diesen beiden Systemen sind keine zu verwendete Hardware und Software gegeben mit denen die Studienarbeit bearbeitet werden muss. Diese Studienarbeit baut auf keiner Studienarbeit oder anderem Projekt auf.

