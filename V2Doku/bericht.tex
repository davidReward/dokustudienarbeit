%%%%%%%%%%%%%%%%%%%%%%%%%%%%%%%%%%%%%%%%%%%%%%%%%%%%%%%%%%%%%%%%%%%%%%%%%%%%%%
%% Descr:       Vorlage für Berichte der DHBW-Karlsruhe
%% Author:      Prof. Dr. Jürgen Vollmer, vollmer@dhbw-karlsruhe.de
%% $Id: bericht.tex,v 1.9 2010/04/10 10:43:57 vollmer Exp $
%%%%%%%%%%%%%%%%%%%%%%%%%%%%%%%%%%%%%%%%%%%%%%%%%%%%%%%%%%%%%%%%%%%%%%%%%%%%%%

\documentclass[
   ngerman          % neue deutsche Rechtschreibung
  ,a4paper          % Papiergrösse
% ,twoside          % Zweiseitiger Druck (rechts/links)
% ,10pt             % Schriftgrösse
% ,11pt
	,12pt
  ,pdftex
]{report}

% Bitte die Codierung Ihrer Dateien auswählen:
%\usepackage[latin1]{inputenc}      % Für UNIX mit ISO-LATIN-codierten Dateien
%\usepackage[applemac]{inputenc}  % Für Apple Mac
%\usepackage[ansinew]{inputenc}   % Für Microsoft Windows
\usepackage[utf8]{inputenc}      % UTF-8 codierte Dateien

% Um Markierungen von Baustellen zu ermöglichen.
\usepackage{color}
\usepackage{bericht}
\usepackage{geometry}
\usepackage{booktabs} % Schönere Tabellen
\usepackage{pdfpages}
\geometry{a4paper, top=25mm, left=25mm, right=25mm, bottom=25mm,
headsep=10mm, footskip=12mm}
\newcommand{\zB}{z.\,B.\ }
\newcommand{\dahe}{d.\,h.\ }

%%%%%%%%%%%%%%%%%%%%%%%%%%%%%%%%%%%%%%%%%%%%%%%%%%%%%%%%%%%%%%%%%%%%%%%%%%%%%%%
%% Angaben zur Arbeit
%%%%%%%%%%%%%%%%%%%%%%%%%%%%%%%%%%%%%%%%%%%%%%%%%%%%%%%%%%%%%%%%%%%%%%%%%%%%%%%

\newcommand{\Autor}{Jan Gerber und David Preiß}
\newcommand{\MatrikelNummer}{5757291 (Jan Gerber) 3199578(David Preiß)}
\newcommand{\Kursbezeichnung}{TINF14B3}

\newcommand{\FirmenName}{EDEKA Handelsgesellschaft Südwest mbH }
\newcommand{\FirmenStadt}{Offenburg }

%\newcommand{\BetreuerFirma}{Roland Schober}
\newcommand{\BetreuerDHBW}{Prof. Hans-Jörg Haubner}

%\newcommand{\Was}{Praxisbericht}
%\newcommand{\Was}{Robotik II}
\newcommand{\Was}{Studienarbeit}
%\newcommand{\Was}{BACHELORARBEIT}

\newcommand{\Titel}{Sensorknoten - Internet der Dinge}

\newcommand{\AbgabeDatum}{15. Mai 2017}


\newcommand{\Dauer}{Theoriephase 5. \& 6. Semester}

\newcommand{\Abschluss}{Bachelor of Engineering}
% \newcommand{\Abschluss}{Bachelor of Science}

\newcommand{\Studiengang}{Informationstechnik}
% \newcommand{\Studiengang}{Angewandte Informatik}

\newcommand{\ITServices}{IT Services }
\newcommand{\ESW}{EDEKA S\"udwest }
\newcommand{\ESWformal}{EDEKA Handelsgesellschaft S\"udwest mbH }


\hypersetup{%%
  pdfauthor={\Autor},
  pdftitle={\Titel},
  pdfsubject={\Was}
}

%%%%%%%%%%%%%%%%%%%%%%%%%%%%%%%%%%%%%%%%%%%%%%%%%%%%%%%%%%%%%%%%%%%%%%%%%%%%%%%

% Wenn \includeonly{..} benutzt wird, werden nur diese Kaptitel ausgegeben.
%\includeonly{
%  abk
% ,kapitel1
% ,kapitel2
% ,changelog
%}

%%%%%%%%%%%%%%%%%%%%%%%%%%%%%%%%%%%%%%%%%%%%%%%%%%%%%%%%%%%%%%%%%%%%%%%%%%%%%%%

% Benutzt man das "biblatex"-Paket, dann muß das hier stehen:
% BIBLATEX
\bibliography{bericht}


\sloppy


% Alter some LaTeX defaults for better treatment of figures:
    % See p.105 of "TeX Unbound" for suggested values.
    % See pp. 199-200 of Lamport's "LaTeX" book for details.
    %   General parameters, for ALL pages:
    \renewcommand{\topfraction}{0.8}	% max fraction of floats at top
    \renewcommand{\bottomfraction}{0.5}	% max fraction of floats at bottom
    %   Parameters for TEXT pages (not float pages):
    \setcounter{topnumber}{2}
    \setcounter{bottomnumber}{1}
    \setcounter{totalnumber}{5}     % 2 may work better
    %\setcounter{dbltopnumber}{2}    % for 2-column pages
    \renewcommand{\dbltopfraction}{0.9}	% fit big float above 2-col. text
    \renewcommand{\textfraction}{0.1}	% allow minimal text w. figs
    %   Parameters for FLOAT pages (not text pages):
    \renewcommand{\floatpagefraction}{0.4}	% require fuller float pages
	% N.B.: floatpagefraction MUST be less than topfraction !!
    \renewcommand{\dblfloatpagefraction}{0.7}	% require fuller float pages

	% remember to use [htp] or [htpb] for placement


\begin{document}
\lstset{language=Java}
\singlespacing % Alles bis zum eigentlichen Text in 1-fachen Zeilenabstand

%%%%%%%%%%%%%%%%%%%%%%%%%%%%%%%%%%%%%%%%%%%%%%%%%%%%%%%%%%%%%%%%%%%%%%%%%%%%%%%
{
\sffamily
\begin{titlepage}
\begin{center}
\vspace*{-2cm}
\hfill
%\includegraphics[width=4cm]{pics/dhbw-logo}\\[2cm]

\begin{figure}
%\subfigure{\includegraphics[width=0.10\textwidth]{pics/edekaneu.png}}\hfill
\subfigure{\includegraphics[width=4cm]{pics/dhbw-logo}}\\[1.5cm]
\end{figure}

{\onehalfspacing \Huge \Titel}\\[1.5cm]

{\Huge  \Was}\\[1.5cm]
{\large für die Prüfung zum}\\[0.5cm]
{\Large \Abschluss}\\[0.5cm]
{\large des Studienganges \Studiengang}\\[0.5cm]
{\large an der}\\[0.5cm]
{\large Dualen Hochschule Baden-Württemberg Karlsruhe}\\[0.5cm]
{\large von}\\[0.5cm]
{\large\bfseries \Autor}\\[1cm]
{\large Abgabedatum \AbgabeDatum}
\vfill


\end{center}
\begin{tabular}{l@{\hspace{2cm}}l}
Bearbeitungszeitraum	         & \Dauer 			\\
Matrikelnummer	                 & \MatrikelNummer		\\
Kurs			         & \Kursbezeichnung		\\
%Ausbildungsfirma				 & \FirmenName \FirmenStadt \\
%Betreuer der Ausbildungsfirma	 & \BetreuerFirma		\\
Gutachter der Dualen Hochschule	 & \BetreuerDHBW		\\
\end{tabular}

\end{titlepage}
}
%%%%%%%%%%%%%%%%%%%%%%%%%%%%%%%%%%%%%%%%%%%%%%%%%%%%%%%%%%%%%%%%%%%%%%%%%%%%%%%

\pagenumbering{roman}

% Nur für Bachelorarbeiten einfügen:
%\newpage
\thispagestyle{empty}
\begin{framed}
\begin{center}
\Large\bfseries \textsf{Sperrvermerk}
\end{center}

\noindent
Die vorliegende Arbeit beinhaltet interne vertrauliche Informationen 
der Firma EDEKA Handelsgesellschaft S�dwest mbH. 


\noindent
Die Weitergabe des Inhaltes der Arbeit und eventuell beiliegender Zeichnungen und Daten im Gesamten oder in Teilen ist grunds�tzlich untersagt. 


\noindent
Es d�rfen keinerlei Kopien oder Abschriften � auch in digitaler Form - gefertigt werden. Ausnahmen bed�rfen der schriftlichen Genehmigung der Firma EDEKA Handelsgesellschaft S�dwest mbH in Abstimmung mit dem/der Verfasser/in.


\noindent
Die vorliegende Arbeit ist nur den Korrektoren sowie ggf. den Mitgliedern des Pr�fungsausschusses zug�nglich zu machen. 



\vspace{2cm}
\noindent
\underline{\hspace{4cm}}\\
Stempel\hspace{4cm}



\vspace{1cm}
\noindent
\underline{\hspace{4cm}}\hfill\underline{\hspace{7cm}}\\
Ort~~~~~Datum\hfill Unterschrift des Ausbildungsleiters\hspace{0.7cm}

\end{framed}
%%%%%%%%%%%%%%%%%%%%%%%%%%%%%%%%%%%%%%%%%%%%%%%%%%%%%%%%%%%%%%%%%%%%%%%%%%%%%%
%% Descr:       Vorlage für Berichte der DHBW-Karlsruhe, Erklärung
%% Author:      Prof. Dr. Jürgen Vollmer, vollmer@dhbw-karlsruhe.de
%% $Id: erklaerung.tex,v 1.1 2010/01/27 17:08:52 vollmer Exp $
%%%%%%%%%%%%%%%%%%%%%%%%%%%%%%%%%%%%%%%%%%%%%%%%%%%%%%%%%%%%%%%%%%%%%%%%%%%%%%%

% In Bachelorarbeiten muss eine schriftliche Erklärung abgegeben werden. In allen anderen
% Arbeiten entfällt diese. Hierin bestätigen die Studierenden, dass die Bachelorarbeit
% selbständig verfasst und sämtliche Quellen und Hilfsmittel angegeben sind. Diese Erklärung
% bildet das zweite Blatt der Arbeit. Der Text dieser Erklärung muss auf einer separaten Seite
% wie unten angegeben lauten.


\newpage
\thispagestyle{empty}
\begin{framed}
\begin{center}
\Large\bfseries \textsf{Erklärung}
\end{center}

\noindent
gemäß § 5 (3) der "`Studien- und Prüfungsordnung DHBW Technik"' vom 22. September 2011. 


\noindent
Ich habe die vorliegende Arbeit selbstständig verfasst und keine anderen als die angegebenen
Quellen und Hilfsmittel verwendet.

\vspace{3cm}
\noindent
\underline{\hspace{4cm}}\hfill\underline{\hspace{6cm}}\\
Ort~~~~~Datum\hfill Unterschrift\hspace{3.85cm}


%\vspace{1cm}
%\noindent
%\underline{\hspace{4cm}}\hfill\underline{\hspace{6cm}}\\
%Ort~~~~~Datum\hfill Unterschrift\hspace{4cm}
%
%
%
%\vspace{1cm}
%\noindent
%\underline{\hspace{4cm}}\hfill\underline{\hspace{6cm}}\\
%Ort~~~~~Datum\hfill Unterschrift\hspace{4cm}

\end{framed}



%%%%%%%%%%%%%%%%%%%%%%%%%%%%%%%%%%%%%%%%%%%%%%%%%%%%%%%%%%%%%%%%%%%%%%%%%%%%%%%
\endinput
%%%%%%%%%%%%%%%%%%%%%%%%%%%%%%%%%%%%%%%%%%%%%%%%%%%%%%%%%%%%%%%%%%%%%%%%%%%%%%%





%\renewcommand{\abstractname}{Zusammenfassung}
%\begin{abstract}
%\begin{onehalfspacing}


%\end{onehalfspacing}
%\end{abstract}

%\newpage
% Beginn römische Ziffern bei den Verzeichnissen

\tableofcontents           % Inhaltsverzeichnis hier ausgeben
\listoffigures             % Liste der Abbildungen
\listoftables              % Liste der Tabellen
\lstlistoflistings         % Liste der Listings
%\listofequations           % Liste der Formeln

%%%%%%%%%% WORKAROUND: Eigener Pagestyle für Abkuerzungsverzeichnis %%%%%%%%%%%%%%
\fancypagestyle{abk}{%
\fancyhf{} % clear all header and footer fields
\fancyfoot[C]{\thepage}
\rhead{\slshape ABKÜRZUNGSVERZEICHNIS}
}
\pagestyle{abk}
%%%%%%%%%%%%%%%%%%%%%%%%%%%%%%%%%%%%%%%%%%%%%%%%%%%%%%%%%%%%%%%%%%%%%%%%%%%%%%%%%%%%%

% Jetzt kommt der "eigentliche" Text
%%%%%%%%%%%%%%%%%%%%%%%%%%%%%%%%%%%%%%%%%%%%%%%%%%%%%%%%%%%%%%%%%%%%%%%%%%%%%%
%% Descr:       Vorlage für Berichte der DHBW-Karlsruhe, Datei mit Abkürzungen
%% Author:      Prof. Dr. Jürgen Vollmer, vollmer@dhbw-karlsruhe.de
%% $Id: abk.tex,v 1.2 2010/01/21 08:40:03 vollmer Exp $
%%%%%%%%%%%%%%%%%%%%%%%%%%%%%%%%%%%%%%%%%%%%%%%%%%%%%%%%%%%%%%%%%%%%%%%%%%%%%%%
\phantomsection %\addcontentsline{toc}{chapter}{Abkürzungsverzeichnis}
\renewcommand\refname{Abkürzungsverzeichnis} \chapter*{Abkürzungsverzeichnis}


%\chapter*{Abkürzungsverzeichnis}                   % chapter*{..} -->   keine Nummer, kein "Kapitel"
						         % Nicht ins Inhaltsverzeichnis
%\addcontentsline{toc}{chapter}{Akürzungsverzeichnis}   % Damit das doch ins Inhaltsverzeichnis kommt

% Hier werden die Abkürzungen definiert
\begin{acronym}[XXXXXXX]	%[XX..] Spaltenabstand in Zeichen vom längsten Acronym
%
\setlength{\itemsep}{-\parsep}	%enger Zeilenabstand

% % % % % % % % % % % % % %
% % Verwendung: \ac{AD} % %
% % % % % % % % % % % % % %


\acro{CSI}{Camera Serial Interface}
\acro{DSI}{Display Serial Interface}
\acro{DSP}{Digitaler Signalprozessor}
\acro{GPIO}{General Purpose Input Output}
\acro{HATEOAS}{Hypermedia As The Engine Of Application State}
\acro{HTTP}{Hypertext Transfer Protoco}
\acro{HTTPS}{HyperText Transfer Protocol Secure}
\acro{IDE}{Integrated Development Environment}
\acro{IoT}{Internet of Things}
\acro{I2C}[I\textsuperscript{2}C]{Inter-Integrated Circuit}
\acro{JSON}{JavaScript Object Notation}
\acro{LXDE}{Lightweight X11 Desktop Environment}
\acro{MMU}{Memory Management Unit}
\acro{NFC}{Near Field Communication}
\acro{PCB}{Printed Circuit Board}
\acro{PIXEL}{Pi Improved Xwindows Environment, Lightweight}
\acro{REST}{Representational State Transfer}
\acro{RFID}{radio-frequency identification}
\acro{RISC}{Reduced Instruction Set Computer}
\acro{ROA}{Resource-oriented Architecture}
\acro{SQL}{Structured Query Language}
\acro{SSL}{Secure Sockets Layer}
\acro{SPI}{Serial Peripheral Interface Bus}
\acro{URI}{Uniform Resource Identifier}
\acro{USART}{Universal synchronous/asynchronous Receiver-Transmitter}
\acro{WDT}{Watchdog Timer}













  
 % \acro{Name}{Darstellung der Abkürzung}{Langform der Abkürzung}
 %\acro{Abk}[Abk.]{Abkürzung}
 
 %müssen noch am Ende händisch (zu Fuß :-) ) sortiert werden.
 
 

\end{acronym}

              % Abkürzungsverzeichnis

%%%%%%%%%%  meinen Pagestyle wiederherstellen %%%%%%%%%%%%%%%
\pagestyle{fancy}
\fancyhf{} % -- alle Bereiche leeren
\rhead{\slshape\leftmark}
\fancyfoot[C]{\thepage}
%%%%%%%%%%%%%%%%%%%%%%%%%%%%%%%%%%%%%%%%%%%%%%%%%%%%%%%%%%%%%


%Beginn arabische Ziffern und 1,5 Zeilenabstand bei eigentlichem Text
\pagenumbering{arabic}

\onehalfspacing

% ########## Gliederung der Arbeit ################


\chapter{Einleitung}

\section{Ziel dieser Arbeit}
-selbstkonfigurierendes Netz
-IoT
-Mesh
-einfache handhabung
-
\section{Ziel dieser Arbeit}
Geplantes vorgehen aus formular
\section{Stand der Technik}
Für die Umsetzung der gestellten Aufgabe eigenen sich Arduino-Mikrocontroller und der Raspberry Pi besonders. Sie lösen die generell bestehenden technischen Probleme relativ gut. Zu diesen Problemen gehört unter anderem der Energieverbrauch und Kommunikation. Für den Bau von Prototypen sind diese Mikrocontroller sehr beliebt. 

Fertige Produkte auf dem Markt nutzen dagegen meist proprietäre eingebettete Systeme. Sie lassen sich dadurch schlecht oder gar nicht modifizieren.   

\section{Motivation und Vorausblick}
Hardwareentwicklung, Softwareentwicklung sowie die Webentwicklung sind die relevanten Bereiche dieser Arbeit. 

Nach der Einleitung betrachten die Autoren die gestellte Problemstellung. 
Im dritten Kapitel werden relevante technische und theoretische Grundlagen mit Hilfe einer Literaturrecherche betrachtet. 

Anschließend folgte die Konzeption. Hier ist die grundsätzlichen Überlegungen zur Realisierung des Projekts niedergeschrieben und graphisch veranschaulicht. 

Im Kapitel Implementierung sind wesentliche Schritte zur Realisierung beschrieben. Relevante Quellcodeausschnitte sind dort erklärt.

Das Fazit betrachtet den Vergleich zwischen der gestellten Aufgabe und dem Ergebnis sowie einen Ausblick inwiefern dieses Projekt fortsetzbar ist.
 
\chapter{Problemstellung}
\section{Ist-Zustand}
Für die Durchführung der Studienarbeit sind bereits zwei Hardwarekomponenten gegeben. Weshalb zunächst auf beide eigesetzten Systeme kurz eingegangen wird.
\paragraph{Raspberry Pi} 
Bei einem Raspberry Pi handelt es sich um einen kleinen Einplatinencomputer. Dieser kann vor allem für kleinere Hard-/Software Projekte eingesetzt werden. Der Raspberry Pi überzeugt vor allem durch  seine Kompaktheit und der Erweiterungsfähigkeit. An den GPIO Schnittstellen können zusätzliche Module für Sensoren, LEDs und Funkmodule angelschlossen werden.  Durch ein integriertes WLAN Modul und einem LAN Modul bietet der Raspberry Pi alle benötigten Komponenten um als Gateway für das Internet der Dinge zu dienen. Der Raspberry Pi kann mit vielen verschiedenen Programmiersprachen programmiert werden, wie zum Beipiel Java, C , C++ oder auch Python. In Kapitel ??????? wird genauer auf den Aufbau des Raspberry Pi’s genauer eingegangen.
\paragraph{Arduino} 
Bei einem Arduino handelt es sich um ein Board mit einem Mikrocontroller. Der Arduino basiert auf einem Atmel-AVR-Mikrocontroller der megaAVR-Serie. Dieser wird mit 5V betrieben und bietet mehrere digitale und analoge  Input- /Output-Pins.  Diese Pins können genutzt werden um verschiedene Sensoren, Aktoren und Funkmodule anzuschließen. Der Arduino kann mit Hilfe des eigens entwickeltem Arduino-IDE per USB Schnittstelle programmiert werden. Die Programmierung erfolgt in C bzw. C++. Der Arduino zeichnet sich vor allem durch seine Kompaktheit und niedrigem Energieverbrauch aus. Diese Eigenschaften machen ihn optimal für Projekte bezüglich Internet der Dinge. In Kapitel ???? wird genauer auf die Eigenschaften und Aufbau des Arduinos eingegangen.

Für die Fertigstellung des Projekts stehen die beiden Theoriesemester des 5. und 6. Semesters des Studiums zum Bachlor of Engineering Informationstechnik bereit. Außer diesen beiden Systemen sind keine zu verwendete Hardware und Software gegeben mit denen die Studienarbeit bearbeitet werden muss. Diese Studienarbeit baut auf keiner Studienarbeit oder anderem Projekt auf.


\section{Soll-Zustand}
Das Ziel der Studienarbeit ist es ein Sensorknoten für das Internet der Dinge zu entwickeln. Hierfür sollen mehrere Arduinos Sensordaten sammeln und diese an einen Raspberry Pi senden. Der Raspberry Pi sendet die gesammelten Daten und sendet sie zu einem externen Server. Der externe Server speichert die aufkommenden Daten in einer Datenbank und stellt sie über eine Webschnittstelle bereit. Die Präsentation der Daten erfolgt über ein Webinterface, auf das der Endbenutzter Zugriff hat. 
\paragraph{Arduino} Der Arduino ist für das Sammeln der Daten zuständig. Er wertet hierfür mehrere Sensoren aus. Die Sensoren sind über verschiedene Bus-Systeme und analogen und digitale Pins angeschlossen. Die Sensoren werten mehrere Eigenschaften aus ihrer Umwelt aus, wie zum Beispiel Temperatur, Luftfeuchtigkeit, Luftdruck, Bewegung und Lichtintensität. 
Die Arduinos nutzten ein Funkmodul um untereinander und mit dem Gateway (Raspberry Pi) zu kommunizieren. Um eine ständige Kommunikation mit dem Raspberry Pi sicherzustellen soll eine Art von Mesh-Netzwerk eingesetzt werden. Dieses erlaubt eine unterbrechungsfreie Kommunikation mit dem Raspberry Pi, falls ein Sensorknoten ausfallen würde.
Um ein praktikablen Einsatz in verschiedenen Umgebungen zu gewährleisten sollten die Arduinos auch teilweise mit Batterien betrieben werden können. Die Arduinos sollten über mehrere Monate hinweg, mit einer Batterieladung, die Daten an den Raspberry senden können.
\paragraph{Raspberry Pi}
Der Raspberry stellt das Gateway zum Internet dar. Er empfängt alle Sensordaten, die mit Hilfe der Arduinos gesammelt wurden. Hierfür nutzt der Raspberry Pi das gleiche Funkmodul, dass auch bei den Arduinos verwendet wurde. Beim Starten des Raspberry Pi führt dieser automatisch die benötigten Programme aus um die Daten zu empfangen. Nachdem der Raspberry Pi die Daten Enkodiert hat sendet er diese an einen externen Server.
\paragraph{Webservice}
Der externe Server besitzt eine SQL Datenbank in der die Sensordaten gespeichert werden. Die Datenbank enthält nur Einträge von zuvor eingetragenen Stationen. Zusätzlich zu den Sensordaten sind zusätzliche Tabellen mit Informationen zu den einzelnen Sensorknoten vorhanden. Mit Hilfe einer REST-Schnittstelle können die Daten als Webservice abgerufen werden. Die Daten sollten zur einfacher Weiterverarbeitung im JSON Datenformat kodiert werden. Die Schnittstelle sollte mehrere Methoden zur Verfügung stellen:
\begin{itemize}
	\item Abruf aller verfügbaren Sensorknoten mit zusätzlichen Informationen, wie zum Beispiel Name und Aufstellungsort
	\item Abruf der neusten Sensordaten eines Sensorknotens
	\item Abruf aller Daten eines Sensor innerhalb eines bestimmten Zeitraumes
\end{itemize}
\paragraph{Webseite} Die Webseite dient zur Präsentation der gesammelten Daten. Der Nutzer soll sich mit Hilfe einer Benutzerkennung und eines Passworts authentifizieren können. Nach einer erfolgreichen Authentifizierung erhält der Nutzer eine Übersicht über alle Stationen die entweder momentan in Betrieb sind oder gegebenenfalls nicht im Betrieb sind. Es sollte zusätzlich die Möglichkeit bestehen, alle aktuellsten Sensorwerte eines einzelnen Sensorknoten anzusehen. Mit Hilfe einer Grafik sollten die Sensordaten über einen Zeitraum anschaubar sein. Der Zeitraum sollte vom Nutzer festgelegt werden können.



\chapter{Theoretische und technische Grundlagen}
\section{Internet of Things}
\section{Arduino}
\section{Raspberry Pi}
\subsection{Hardware}
Hier


\chapter{Konzeption}
\section{Mesh-Netz}
\label{sec:KonzeptionMeshNetz}
Für die Entwicklung eines Mesh-Netzes muss zunächst eine Topologie designt werden, die den Eigenschaften der Sensorknoten entsprechen. In das Netzwerk muss zu einem ein Sensorknoten eingebunden werden der extrem Energiesparend ist, ein normaler Sensorknoten der über eine dauerhafte Stromversorgung verfügt und ein Gateway zum Internet. Jeder dieser Geräte hat unterschiedliche Anforderungen und teilweise auch verschiedene Betriebssysteme. 
\begin{figure}
	\centering
	\includegraphics[width=0.6\textwidth]{bilder/konzeptionMeshTopologie.png}
	\caption[Topologie Mesh-Netz]{Topologie Mesh-Netz: blaue Knoten $\rightarrow$ energiesparende Sensorknoten, grüne Knoten $\rightarrow$ normale Sensorknoten und gelber Knoten $\rightarrow$ Gateway}
	\label{img:konzeptionTopologie}
\end{figure}

In der Konzeptionsphase wurde die Geräte in drei Gruppen unterteilt, diese waren die normale Sensorknoten, energiesparende Sensorknoten und ein Gateway zum Datenbankserver/Internet. In der Grafik \ref{img:konzeptionTopologie} ist die Topologie des Mesh-Netzes zu sehen.
\paragraph{Normaler Sensorknoten} Dieser Sensorknoten übernimmt neben der Messwerterfassung noch das Weiterleiten von Nachrichten im Netztwerk. Hierfür ist er mit einer dauerhaften Stromversorgung ausgestattet. Während der Sendephase des Sensorknotens kann dieser keine Nachrichten empfangen und weiterleiten. Wenn der Sensorknoten nicht in der Sendephase ist hört er die gesamte Zeit in das Netzwerk und wartet bis er eine Nachricht empfängt. Der genaue Ablauf wird im darauffolgenden Paragraphen genauer erläutert. Wie in den Grundlagen (\ref{sec:Mesh-Netz} Kapitel) erläutert handelt es sich um einen aktiven Knoten, genauer um einen Mesh-Router. Mit diesen Geräten kann das Netz, bezüglich Reichweite, erweitert werden.
\paragraph{Energiesparender Sensorknoten} Beim energiesparenden Sensorknoten handelt es sich nur um einen Mesh-Client (siehe \ref{sec:Mesh-Netz} Kapitel), dass bedeutet er ist ein passiver Knoten im Mesh-Netz. Dieser übernimmt keine Weiterleitungsaufgaben im Netzwerk. Nachdem der Sensorknoten seine Messwerte erfasst und gesendet hat geht dieser Knoten in einen Schlafmodus. Mit diesem Sensorknoten ist es nicht möglich die Reichweite des Netzwerks zu erhöhen. Dieser Knoten ist darauf angewiesen, dass entweder ein normaler Sensorknoten in der Nähe ist oder direkt das Gateway.
\paragraph{Gateway zum Internet} Beim der letzten Komponente des Mesh Netzes handelt es sich um das Gateway zum Internet. Dieses Modul übernimmt nur das Empfangen von Nachrichten. Das Gateway ist im permanenten Empfangsmodus und wartet auf das Eingehen neuer Nachrichten. Nachdem das Datenpaket enkodiert ist wird sichergestellt, dass diese Nachricht zum ersten Mal angekommen ist. Wenn die Nachricht zum ersten Mal erhalten wurde, wird der Messdatensatz dauerhaft gespeichert. 
\paragraph{Programmablaufplan Mesh-Algorithmus} Nachdem wir die einzelnen Geräte genauer beleuchtet haben wird in diesem Abschnitt auf den selbst Entwickelten Art von Mesh-Algorithmus eingegangen. In der Grafik \ref{img:PAPMeshAlgo} ist der Programmablaufplan bei einem normalen Sensorknoten zu sehen. Dieser beschreibt das Vorgehen beginnend vom Empfang einer Nachricht bis zum Verwurf der Nachricht bzw. Weiterleiten der Nachricht. 

\begin{figure}
	\centering
	\includegraphics[width=0.6\textwidth]{bilder/PAPMesh.png}
	\caption{Programmablaufplan - Mesh- Algorithmus}
	\label{img:PAPMeshAlgo}
\end{figure}

\section{Anbindung Pi/Arduino}
-RF24
-Kodierung
\section{Messwerterfassung mit Arduino}
-Ablauf der Messwerterfassung
-Vorstellen der Messgrößen
-Eagle vorstellen

\chapter{Realisierung/Implementierung}
\section{Topologie}
-zuordnung aktiv/passiv
-schema mit bildern
-raspi anbindung an web
-code mesh netz
-Anpassung an lib für broadcast

\section{Verwendete Sensoren}
-Code von Sensor-Abfrage --> welche Sensoren(DHT11 etc)
\section{Hardware Design Sensorknoten}
\label{sec:HardwareDesign}
In diesem Unterkapitel wird das Design des Sensorknotens, sowie der Produktionsprozess näher betrachtet. Es wird dabei zuerst auf den energiesparenden Sensorknoten, der mit dem Arduino Pro Mini realisiert wurde, und anschließend auf den normalen Sensorknoten, der mit dem Arduino Nano eingegangen.
\subsection{Arduino Pro Mini}
\begin{figure}
	\centering
	\includegraphics[width=0.6\textwidth]{bilder/mini_cutted.jpg}
	\caption[Energiesparender Sensorknoten]{Energiesparender Sensorknoten: bestückt mit verschiedenen Sensoren und einem Arduino Pro Mini}
	\label{img:ArduinoProMini}
\end{figure}
Der energiesparende Sensorknoten wird dauerhaft mit Batterien betrieben, dieser ist in Bild \ref{img:ArduinoProMini} zu sehen. Der Sensorknoten ist mit einem Arduino Pro Mini bestückt, da dieser energiesparender ist als der Arduino Nano. Bei Sensorknoten lassen sich mit den in Kapitel \ref{sec:VerwendeteSensoren} vorgestellten Sensoren bestücken. Die Sensoren können mit Hilfe von Stiftleisten jederzeit getauscht werden, nur beim DHT22 bzw. DHT11 haben wir uns entschieden diesen direkt aufzulöten. Der Grundgedanke war dabei, dass jeder Sensorknoten als Mindestanforderung die Temperatur und Luftfeuchtigkeit bestimmen kann. Jeder Sensorknoten besitzt ein nRF24L01 Modul zur Kommunikation mit anderen Sensorknoten oder dem Raspberry Pi.
\paragraph{Spannungsversorgung} Der Arduino Pro Mini verfügt über kein 5V zu 3,3V Wandler, weshalb zusätzlich noch ein Wandler aufgebracht wurde. Dieser 5V zu 3,3V Wandler ist ebenfalls nur mit Hilfe von Stiftleisten aufgebracht. 

Die Spannungsversorgung des Sensorknoten erfolgt mit 4 x 1,5 AA Batterien, diese liefern gemeinsam eine Spannung von 6V zu Beginn ihrer Betriebszeit. Der Sensorknoten kann bis ca. 4,2V betrieben werden. Die Spannungszufuhr kann mit Hilfe eines Schalters unterbrochen werden. 
\paragraph{Aufgetretene Probleme} Bei der Erstellung des Schaltplans ist ein Fehler bei Sensorschnittstellen für den BMP180 und BH1750 entstanden. Beide Sensoren werden über das $I^2C$ Schnittstelle angesprochen. Hierbei kam es zu einer Vertauschung der 3,3V Leitung und der Masse Leitung. Aus diesem Grund können diese beiden Sensoren nur mit einem Adapter bzw. mit flexiblen Steckbrücken betrieben werden. Die Funktion der Schnittstelle sind davon nicht betroffen. 

Zusätzlich kam es bei der Auswertung der Reed-Kontakte zu zufälligen falschen Werten. Diese konnte behoben werden in dem ein zusätzlicher 10k Ohm Pull Down-Widerstand eingelötet wurde, der das Signal auf Masse zieht.

Um zusätzlich zufällige Werte auszuschließen, wenn kein Bewegungsmelder oder Bodenfeuchtigkeitsmesser angeschlossen ist, wurde eine Steckbrücke genutzt. Diese Steckbrücke verbindet die Datenleitung mit Masse.  Durch dieses Verfahren kann einfach überprüft werden ob die Sensoren angeschlossen sind oder nicht.
\label{sec:ArduinoProMini}
\subsection{Arduino Nano}
\label{sec:ArduinoNano}
\begin{figure}
	\centering
	\includegraphics[width=0.6\textwidth]{bilder/SensorknotenArduinoNano.jpg}
	\caption[Normaler Sensorknoten]{Normaler Sensorknoten: bestückt mit verschiedenen Sensoren und einem Arduino Nano}
	\label{img:ArduinoNano}
\end{figure}
Der normale Sensorknoten ist mit einem Arduino Nano bestückt. Dieser verfügt über die gleichen Anschlussmöglichkeiten wie der energiesparende Sensorknoten. In Bild \ref{img:ArduinoNano} ist der normale Sensorknoten zu sehen.
\paragraph{Spannungsversorgung} Der normale Sensorknoten kann entweder über die vorhandene USB Schnittstelle betrieben werden oder ebenfalls wie der Arduino Pro Mini über eine externe Stromversorgung, wie zum Beispiel Batterien oder Akkus. Da diese Sensorknoten jedoch deutlich mehr Energie benötigt als der energiesparende Sensorknoten, sollten genügend mAh zur Verfügung stellen. Die externe Stromversorgung kann über ein Schalter ausgeschalten werden. Der Arduino Nano verfügt direkt auf dem Board ein Wandler von 5V zu 3,3V.
\paragraph{Aufgetretene Probleme} Mit dem normalen Board sind die gleichen Probleme hinsichtlich dem Einsatz von Widerständen aufgetreten, diese Probleme konnten wie beim energiesparende Sensorknoten gelöst werden. Das Sensorknoten Board war allerdings vollständig richtig, was das aufbringen und wechseln von Sensoren deutlich erleichtert.
\subsection{Produktionsprozess der Sensorknoten}
\label{sec:ProduktionsprozessSensornoten}
In diesem Unterkapitel wird auf den gesamten Produktionsprozess eingegangen. Beginnend mit der Entwicklung eines Prototypens, über die Erstellung genauer Schaltpläne und abschließend mit der Bestellung der Platinen, sowie der Bestückung dieser Platinen.
\paragraph{Erstellung eines Prototypen} Um sich in das Thema einzuarbeiten, wurde zunächst ein Prototyp entwickelt ( siehe Bild \ref{img:prototyp}. Dieser wurde auf einer einfachen Lochrasterplatine entwickelt. Die einzelnen Komponenten wurden mit Drahtstücken verbunden. Der Löt- und Bestückungsvorgang ging pro Prototyp 2-3 Stunden. Auf dem Prototyp war das Funkmodul, der DHT22 Sensor und eine Sensor mit einer $I^2C$ Schnittstelle aufgebracht. Alle Sensoren konnten in Steckleisten befestigt werden, dies ermöglichte einen schnellen Austausch falls ein Sensor ausfallen würde. Für die Erstellung der Prototypen wurde ein handschriftlicher Schaltplan entworfen.
\begin{figure}
	\centering
	\includegraphics[width=0.6\textwidth]{bilder/prototyp.jpg}
	\caption[Prototyp Sensorknoten]{ Prototyp Sensorknoten bestückt mit einem Arduino Nano, DHT22 und dem nRF24l01 Funkmodul und wahlweise mit einem BH1750 oder BMP180}
	\label{img:prototyp}
\end{figure}

\paragraph{Erstellung eines Schaltplans} Nachdem ein funktionsfähiger Prototyp entworfen wurde, konnte ein Schaltplan für den energiesparenden Sensorknoten und den normalen Sensorknoten entwickelt werden. Die Schaltpläne beider Sensorknoten sind im Anhang zu finden. Der Schaltplan wurde komplett entworfen ohne alle Bauteile getestet zu haben. Zusätzlich wurde nur für den normalen Sensorknoten ein Prototypen entworfen, da zum Zeitpunkt der Erstellung des Prototyen die Arduino Pro Mini nicht vorlagen. Der Schaltplan wurde mit Hilfe von Eagle (siehe Kapitel \ref{sec:MesswerterfassungArduino} erstellt.
\paragraph{Erstellung eines Layouts} Eagle bietet die Möglichkeit, nach der Erstellung eines Schaltplans ein Layout für diesen Schaltplan zu entwickeln. Zunächst muss die Größe der Platine festgelegt werden, diese ist bei den Sensorknoten 70 mm * 69 mm. Im nächsten Schritt werden alle Bauteile auf der Platine platziert und positioniert. Nachdem die Bauteile so positioniert worden sind, dass genügend Platz zwischen den Bauteilen ist. Können die Leiterbahnen Routen, mit Hilfe des Autorouters auf der Leiterplatte verlegt werden. Die Leiterbahnen werden nach bereits vorher angegeben Anforderungen (zum Beispiel Abstand und Breite Leiterbahnen) verlegt. Im letzten Schritt wird die Platine noch beschriftet und Bohrlöcher in den Ecken hinzugefügt. So besteht auch die Möglichkeit die Platinen in einem Gehäuse zu befestigen. 
\paragraph{Bestellprozess} Da das Erstellen der Platinen Zeitaufwendig wäre haben wir uns entschieden die Platinen produzieren zu lassen. Nach einem kurzen Vergleich verschiedener Hersteller, 
\paragraph{Bestückung de Platinen}

%\input{kapitel/05/sender}
\section{Raspberry Pi}
\subsection{Verkabelung}
-logisch
\subsection{Receiver}
-Code
\subsection{Kodierung}
\subsection{Hashing}
-Code
\subsection{Datenbankanbindung}
-Code
\subsection{Scheduling}
-SystemD job
%\input{kapitel/05/dv}
\section{Webservice (mit Flask)}
-Routen von Webservice
\section{Datenvisualisierung}
\label{sec:Datenvisualiserung}
Ein weiteres Ziel der Studienarbeit war es die gesammelten Daten zu präsentieren. Die Entscheidung fiel auf eine Webapplikation. Der Grund für diese Entscheidung war, dass die Anwendung von vielen verschiedenen Geräten genutzt werden kann. 

Als Technologie wurde AngularJS gewählt, ein clientseitiges JavaScript Webframework zur Erstellung von Singel-Page-Webanwendungen. Zur Präsentation der Webseite wurde HTML5 und CSS3 eingesetzt. Zur Oberflächengestaltung wurde Bootstrap (CSS-Framework) eingesetzt.

Die Webseite besteht aus drei Seiten insgesamt. Auf diese kann erst nach einer Authentifizierung zugegriffen werden. Die Webseite nutzt den in Kapitel \ref{sec:webservice} vorgestellten REST Webservice, um auf die Daten aus der Datenbank zuzugreifen.
\begin{figure}
	\centering
	\includegraphics[width=0.6\textwidth]{bilder/WebseiteUebersicht}
	\caption{Webseite: Übersicht aller Sensorknoten}
	\label{img:WebseiteUebersicht}
\end{figure}

\paragraph{Übersicht aller Sensorknoten} Zunächst erhält der Nutzter, wie in Bild \ref{WebseiteUebersicht}, eine Übersicht über alle Sensorknoten. Die Übersicht enthält dabei alle Sensorknoten, die bereits einmal einen Datensatz in der Datenbank gespeichert haben.

Die Daten werden in einer Tabelle dargestellt, diese enthält dabei Informationen zur Stations Nummer, Name der Station, der Ort an dem das Gerät sich befindet und ob die Station sich im Energiesparmodus befindet oder nicht. Von dieser Übersichtsseite gelangt der Nutzer mit dem Button \textit{Aktuelle Werte} zu den aktuellen Werten des Sensorknotens.
\begin{figure}
	\centering
	\includegraphics[width=0.7\textwidth]{bilder/WebseiteSensorknotenAktuell}
	\caption{Webseite: Aktuelle Übersicht eines Sensorknotens}
	\label{img:WebseiteSensorknotenAktuell}
\end{figure}
\paragraph{Aktuelle Werte eines Sensorknotens} Auf dieser Seite kann sich der Nutzere die aktuellsten Werte des Sensorknotens angezeigt werden lassen. Die Anzahl der Kacheln wird dynamisch Erzeugt, dies ist Abhänigig davon welche Messwerte der Sensorknoten bereits gesendet hat. In einer Kachel ist enthalten wann der Messwert gemessen wurde, welchen Wert der Messwert hat und welche Einheit dieser hat. Zusätzlich ist noch ein Button auf jeder Kachel bei dem sich der Nutzer noch die Grafik der vergangene Werte anzeigen lassen kann. Die Webseite aktualisiert die Daten alle 30 Sekunden, es besteht aber auch die Möglichkeit(Button oben rechts) eine manuelle Aktualisierung anzustoßen.
\begin{figure}
	\centering
	\includegraphics[width=0.7\textwidth]{bilder/WebseiteGrafik}
	\caption{Webseite: Grafik eines Messwertedatensatz}
	\label{img:WebseiteGrafik}
\end{figure}

\paragraph{Grafik über einen bestimmten Zeitraum} Zur Ansicht älterer Sensorwerte besteht die Möglichkeit sich die gesammelten Werte in einem Kurvendiagramm darzustellen. Der Nutzer muss zunächst die Station und den Sensorwert auswählen, den er betrachten möchte. Anschließend muss er angeben in welchem Zeitraum er die Messwerte dargestellt haben möchte. Mit dem Betätigen des Buttons Grafik anzeigen wird die Grafik angezeigt. In der x-Achse befindet sich das Datum, in der y-Achse ausgewählte Wert. Die x- und y-Achse wird entsprechend der gewählten Größen angepasst. Mit der Maus kann über die Kurve gefahren werden und sich mit Hilfe eines Tooltipps genauere Informationen angezeigt werden lassen. Für die Grafikanzeige wurde ein JavaScript Plugin verwendet, mit dem verschiedene Diagrammtypen dargestellt werden können. Als Framework wurde \textit{Highcharts} verwendet. Momentan werden noch alle Werte angezeigt, hier könnte bei einer großen Anzahl an Messwerten eine Reduzierung / Ausdünnung vorgenommen werden.


\section{Datenbankschema}
Zu Beginn der Entwicklung kam \textit{SQLite} als Datenbankmanagementsystem zum Einsatz. Es stellte sich jedoch heraus, dass es zu Problemen bei gleichzeitigem Zugriff auf die Datenbank kommt. Daher kommt momentan \textit{MySQL} zum Einsatz. MySQL ist weit verbreitetes Datenbankmanagementsystem, ist es Open Source und frei verfügbar. Mit mehreren Datenbankverbindungen kommt es problemlos aus. Das System wird auf dem gleichen Host wie der Webserver für die Datenrepräsentation und der Webservice für die Datenbereitstellung ausgeführt. Dieser externe Server läuft in einer virtuellen Maschine der DHBW Karlsruhe. Das Betriebssystem des Server ist ein Ubuntu-Server. 
\begin{figure}[h!]
\includegraphics[scale=0.8]{bilder/EERDiagramm} 
\caption{Schema der Datenbank}
\label{Datenbankschema}
\end{figure}   
Die Datenbank beinhaltet drei Tabellen. Bei dem Attribut \textit{id} vom Typ CHAR(32) in der Tablle \textit{messwerte} handelt es sich um die gehashte ID. Diese Tabelle enthält die eigentlichen Messdaten. Die ID dient als Primarykey. Das Attribut \textit{originAddr} gibt an von welchem Sensorknoten (Arduino) das Tupel stammt. \textit{value} gibt den Messwert der entsprechenden Messgröße beziehungsweise Einheit (\textit{unit}) an. Der \textit{timestamp} ist im UNIX-Format. Er zählt die Sekunden seit dem ersten 01.01.1970 hoch. Diese Datumsrepräsentation ist besonders simpel. Da die Arduinos keine Echtzeituhr haben, erstellt der Raspberry Pi beim Empfang den Zeitstempel.
Die Attribute \textit{originAddr} und \textit{unit} sind in \textit{messwerte} sind jeweils Fremdschlüssel. \textit{originAddr} ist Primärschlüssel in der Tabelle \textit{stationen}. Sie dient der Speicherung aller Sensorknoten. Die Tabelle enthält außerdem noch die Spalte \textit{name}, den Standort (\textit{location}) und ein Flag (\textit{powerSavig}), ob es sich um einen energiesparenden Arduino handelt oder nicht.
\textit{einheiten} hat als eindeutigen Primärschlüssel \textit{unit\_id}. In der Tabelle sind alle Messgrößen enthalten. Ein String (VARCHAR(32)) bezeichnet den Sensor (zum Beispiel Luftfeuchte). \textit{unit\_name} repräsentiert die Einheit der Messgröße (zum Beispiel \textit{\%}) 

\section{Energiesparmodus Arduino Pro Mini}
\label{sec:Energiesparmodus}
In diesem Unterkapitel wird auf die eingesetzten Techniken und Anpassungen eingegangen. Diese sind notwendig, um einen sehr energiesparenden Sensorknoten zu entwickeln. Der Sensorknoten sollte wie bereits beschrieben nur mit 4x AA-Batterien betrieben werden. Hierfür wurden drei grundlegende Ideen umgesetzt. Diese werden in den folgenden Abschnitten genauer erläutert. 
\paragraph{Verwendung des Arduino Pro Mini} Der Arduino Pro Mini eignet sich besonders für Projekte bei denen nur wenig Energie verbraucht werden darf. Der Pro Mini verfügt im Gegensatz zum Arduino Nano über keinen USB Anschluss und keinen Spannungswandler von 5V zu 3,3V. Diese beiden Komponenten benötigen zusätzlich Strom. Durch den Verzicht auf diese beiden Komponenten kann ca. 2 - 4mA eingespart werden.
\paragraph{Entfernen der Status LED} Auf den Arduinos ist eine Status LED aufgebracht. Diese zeigt nur an, ob der Arduino aktiv ist oder nicht. Da diese LED keine weitere Funktion hat kann sie entfernt werden. Das Löten von SMD-LEDs ist extrem schwierig. Aus diesem Grund wird nicht die LED selbst ausgelötet, sondern der verbaute Vorwiderstand. So kann gegebenenfalls die LED einfacher wieder in Betrieb genommen werden. Der Ausbau der LED bringt eine Ersparnis von 3 – 4 mA.
\paragraph{Verwendung des  Watchdog-Timer und Sleep Modus} Um jedoch wirklich Energie zu sparen muss der Mikrocontroller in einen Schlafmodus versetzt werden. Hierfür wird der Watchdog-Timer verwendet (siehe Kapitel \ref{sec:AvrMikrocontroller}). Der Watchdog Timer wird verwendet um den Mikrocontroller 8 Sekunden in den Schlafmodus zu versetzten. In dieser Zeit verbraucht der Mikrocontroller fast keinen Strom mehr, da dieser in der Zeit keine Programmabarbeitungen vornimmt. Die Umsetzung dieser Funktion wurde mit Hilfe einer \textit{Low-Power} Bibliothek (https://github.com/rocketscream/Low-Power) umgesetzt.
\paragraph{} In Tabelle \ref{tbl:Energiesparen} sind die Einsparungen der verschiedenen Techniken zu sehen. Bei der Messung war der DHT22, BH1750 und das Funkmodul angeschlossen. Am besten hat die Kombination aus einem Arduino Pro Mini ohne LEDs im Sleep-Modus mit nur 182 $\mu$A  abgeschnitten. Mit diesen Techniken kann der energiesparende Sensorknoten über mehrere Monate hinweg über Batterie betrieben werden. 
\begin{table}[]
	\centering
	\caption{Stromstärken bei den verschiedenen Energiespartechniken}
	\label{tbl:Energiesparen}
	\begin{tabular}{l|l|l}
		& Normaler Modus   & Sleep Modus \\ \hline
		Arduino Nano              & 19,61 - 29,31 mA & 7,62 mA     \\ \hline
		Arduino Pro Mini          & 17,26 - 28,28 mA & 3,24 mA     \\ \hline
		Arduino Nano ohne LED     & 15,71 - 26,20 mA & 3,09 mA     \\ \hline
		Arduino Pro Mini ohne LED & 14,96 - 24,83 mA & 182 $\mu$A     
	\end{tabular}
\end{table}





\chapter{Lösungsbewertung und Fazit}

 

% #################################################



%\include{changelog}

\singlespacing %Literaturverzeichnis wieder 1-facher Zeilenabstand.

\addcontentsline{toc}{chapter}{Literaturverzeichnis}

% Haben Sie das "biblatex"-Paket nicht installiert, benutzen Sie folgendes:
% Ohne das "biblatex"-Paket (s. bericht.sty) produziert folgendes
% "deutsche" Zitate in Literaturverzeichnissen gemaß der Norm DIN 1505,
% Teil 2 vom Jan. 1984.
% Die Zitatmarken werden alphabetisch nach Verfassern
% sortiert und sind durch abgekürzte Verfasserbuchstaben plus
% Erscheinungsjahr in eckigen Klammern gekennzeichnet.

% \bibliographystyle{alphadin}
% \bibliography{bericht}

%%%%%%%%%%%%%%%%%%%%%%%%%%%%%%%%%%%%%%%5
% BIBLATEX
% Benutzt man das "biblatex"-Paket, muß man folgendes schreiben:
\def\refname{Literaturverzeichnis}
\printbibliography    

%%%%%%%%%%%%%%%%%%%%%%%%%%%%%%%%%%%%%%%5

% Ab hier beginnt der Anhang


\includepdf[pages=-]{bilder/ArduinoNano-rotated.pdf}
\includepdf[pages=-]{bilder/ArduinoProMini-rotated.pdf}
\includepdf[pages=-]{bilder/ArduinoProMiniUeberarbeitet-rotated.pdf}

\newpage
%\addcontentsline{toc}{chapter}{Liste der ToDo's}
%\listoftodos[Liste der ToDo's]



\end{document}

