\section{Internet of Things}
\ac{IoT} steht für die Vernetzung von Gegenständen, die primär kein Rechner sind. \ac{IoT} ist in der Literatur nicht scharf definiert. Der Wissenschaftliche Mitarbeiter des MITs Kevin Ashton hat sich 1999 wie folgt zum Thema \ac{IoT} geäußert: 
\begin{quote}
If we had computers that knew everything there was to know about things -- using data they gathered 
without any help from us -- we would be able to track and count everything, and greatly reduce waste, 
loss and cost. We would know when things needed replacing, repairing or recalling, and whether they 
were fresh or past their best. We need to empower computers with their own means of gathering 
information, so they can see, hear and smell the world for themselves, in all its random glory. RFID and 
sensor technology enable computers to observe, identify and understand the world—without the 
limitations of human-entered data.
\end{quote}

%TODO Definition Sensorknoten
%TODO Quelle: http://www.cisco.com/c/dam/en_us/solutions/trends/iot/introduction_to_IoT_november.pdf
