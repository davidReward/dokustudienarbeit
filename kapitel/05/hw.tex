\section{Hardware Design Sensorknoten}
\label{sec:HardwareDesign}
In diesem Unterkapitel werden das Design des Sensorknotens sowie der Produktionsprozess näher betrachtet. Es wird dabei zuerst auf den energiesparenden Sensorknoten, der mit dem Arduino Pro Mini realisiert wurde und anschließend auf den normalen Sensorknoten, der mit dem Arduino Nano eingegangen.
\subsection{Arduino Pro Mini}
\begin{figure}
	\centering
	\includegraphics[width=0.6\textwidth]{bilder/mini_cutted}
	\caption[Energiesparender Sensorknoten]{Energiesparender Sensorknoten: bestückt mit verschiedenen Sensoren und einem Arduino Pro Mini}
	\label{img:ArduinoProMini}
\end{figure}
Der energiesparende Sensorknoten wird dauerhaft mit Batterien betrieben. Dieser ist in Bild \ref{img:ArduinoProMini} zu sehen. Der Sensorknoten ist mit einem Arduino Pro Mini bestückt, da dieser energiesparender ist als der Arduino Nano. Sensorknoten lassen sich mit den in Kapitel \ref{sec:VerwendeteSensoren} vorgestellten Sensoren bestücken. Die Sensoren können mit Hilfe von Stiftleisten jederzeit getauscht werden. Nur beim DHT22 bzw. DHT11 haben wir uns entschieden diesen direkt aufzulöten. Der Grundgedanke dabei war, dass jeder Sensorknoten als Mindestanforderung die Temperatur und Luftfeuchtigkeit bestimmen kann. Jeder Sensorknoten besitzt ein nRF24L01 Modul zur Kommunikation mit anderen Sensorknoten oder dem Raspberry Pi.
\paragraph{Spannungsversorgung} Der Arduino Pro Mini verfügt über kein 5V zu 3,3V Wandler, weshalb zusätzlich noch ein Wandler aufgebracht wurde. Dieser 5V zu 3,3V Wandler ist ebenfalls nur mit Hilfe von Stiftleisten aufgebracht. 

Die Spannungsversorgung des Sensorknoten erfolgt mit 4 x 1,5 AA Batterien. Diese liefern gemeinsam eine Spannung von 6V zu Beginn ihrer Betriebszeit. Der Sensorknoten kann bis ca. 4,2V betrieben werden. Die Spannungszufuhr kann mit Hilfe eines Schalters unterbrochen werden. 
\paragraph{Aufgetretene Probleme} Bei der Erstellung des Schaltplans ist ein Fehler bei Sensorschnittstellen für den BMP180 und BH1750 entstanden. Beide Sensoren werden über das $I^2C$ Schnittstelle angesprochen. Hierbei kam es zu einer Vertauschung der 3,3V Leitung und der Masse Leitung. Aus diesem Grund können diese beiden Sensoren nur mit einem Adapter bzw. mit flexiblen Steckbrücken betrieben werden. Die Funktion der Schnittstelle ist davon nicht betroffen. 

Zusätzlich kam es bei der Auswertung der Reed-Kontakte zu zufälligen falschen Werten. Diese konnten behoben werden in dem ein zusätzlicher 10k Ohm Pull Down-Widerstand eingelötet wurde, der das Signal auf Masse zieht.

Um zusätzlich zufällige Werte auszuschließen, wenn kein Bewegungsmelder oder Bodenfeuchtigkeitsmesser angeschlossen ist, wurde eine Steckbrücke genutzt. Diese Steckbrücke verbindet die Datenleitung mit Masse. Durch dieses Verfahren kann einfach überprüft werden ob die Sensoren angeschlossen sind oder nicht.
\label{sec:ArduinoProMini}
\subsection{Arduino Nano}
\label{sec:ArduinoNano}
\begin{figure}
	\centering
	\includegraphics[width=0.6\textwidth]{bilder/SensorknotenArduinoNano}
	\caption[Normaler Sensorknoten]{Normaler Sensorknoten: bestückt mit verschiedenen Sensoren und einem Arduino Nano}
	\label{img:ArduinoNano}
\end{figure}
Der normale Sensorknoten ist mit einem Arduino Nano bestückt. Dieser verfügt über die gleichen Anschlussmöglichkeiten wie der energiesparende Sensorknoten. In Bild \ref{img:ArduinoNano} ist der normale Sensorknoten zu sehen.
\paragraph{Spannungsversorgung} Der normale Sensorknoten kann entweder über die vorhandene USB Schnittstelle betrieben werden oder ebenfalls wie der Arduino Pro Mini über eine externe Stromversorgung, wie zum Beispiel Batterien oder Akkus. Da diese Sensorknoten jedoch deutlich mehr Energie benötigt als der energiesparende Sensorknoten, sollten genügend mAh zur Verfügung gestellt werden. Die externe Stromversorgung kann über einen Schalter ausgeschalten werden. Der Arduino Nano verfügt direkt auf dem Board ein Wandler von 5V zu 3,3V.
\paragraph{Aufgetretene Probleme} Mit dem normalen Board sind die gleichen Probleme hinsichtlich dem Einsatz von Widerständen aufgetreten. Diese Probleme konnten wie beim energiesparende Sensorknoten gelöst werden. Das Sensorknoten Board war allerdings vollständig richtig, was das aufbringen und wechseln von Sensoren deutlich erleichtert.
\subsection{Produktionsprozess der Sensorknoten}
\label{sec:ProduktionsprozessSensornoten}
In diesem Unterkapitel wird auf den gesamten Produktionsprozess eingegangen. Beginnend mit der Entwicklung eines Prototypens, über die Erstellung genauer Schaltpläne und abschließend mit der Bestellung der Platinen, sowie der Bestückung dieser Platinen.
\paragraph{Erstellung eines Prototypen} Um sich in das Thema einzuarbeiten, wurde zunächst ein Prototyp entwickelt (siehe Bild \ref{img:prototyp}. Dieser wurde auf einer einfachen Lochrasterplatine entwickelt. Die einzelnen Komponenten wurden mit Drahtstücken verbunden. Der Löt- und Bestückungsvorgang ging pro Prototyp 2-3 Stunden. Auf dem Prototyp war das Funkmodul, der DHT22 Sensor und eine Sensor mit einer $I^2C$ Schnittstelle aufgebracht. Alle Sensoren konnten in Steckleisten befestigt werden. Dies ermöglichte einen schnellen Austausch, falls ein Sensor ausfallen würde. Für die Erstellung der Prototypen wurde ein handschriftlicher Schaltplan entworfen.
\begin{figure}
	\centering
	\includegraphics[width=0.6\textwidth]{bilder/prototyp}
	\caption[Prototyp Sensorknoten]{ Prototyp Sensorknoten bestückt mit einem Arduino Nano, DHT22 und dem nRF24l01 Funkmodul und wahlweise mit einem BH1750 oder BMP180}
	\label{img:prototyp}
\end{figure}

\paragraph{Erstellung eines Schaltplans} Nachdem ein funktionsfähiger Prototyp entworfen wurde, konnte ein Schaltplan für den energiesparenden Sensorknoten und den normalen Sensorknoten entwickelt werden. Die Schaltpläne beider Sensorknoten sind im Anhang zu finden. Der Schaltplan wurde komplett entworfen ohne alle Bauteile getestet zu haben. Zusätzlich wurde nur für den normalen Sensorknoten ein Prototypen entworfen, da zum Zeitpunkt der Erstellung des Prototyen die Arduino Pro Mini nicht vorlagen. Der Schaltplan wurde mit Hilfe von Eagle (siehe Kapitel \ref{sec:MesswerterfassungArduino} erstellt.
\paragraph{Erstellung eines Layouts} Eagle bietet die Möglichkeit, nach der Erstellung eines Schaltplans ein Layout für diesen Schaltplan zu entwickeln. Zunächst muss die Größe der Platine festgelegt werden, diese ist bei den Sensorknoten 70 mm * 69 mm. Im nächsten Schritt werden alle Bauteile auf der Platine platziert und positioniert. Die Bauteile wurden so positioniert, dass genügend Platz zwischen den Bauteilen ist.  Anschließend können die Leiterbahnen, mit Hilfe des Autorouters auf der Leiterplatte verlegt werden. Die Leiterbahnen werden nach bereits vorher angegeben Anforderungen (zum Beispiel Abstand und Breite Leiterbahnen) verlegt. Im letzten Schritt wird die Platine noch beschriftet und Bohrlöcher in den Ecken hinzugefügt. So besteht auch die Möglichkeit die Platinen in einem Gehäuse zu befestigen. 
\paragraph{Bestellprozess} Da das Erstellen der Platinen Zeitaufwendig wäre haben wir uns entschieden die Platinen produzieren zu lassen. Nach einem kurzen Vergleich verschiedener Hersteller kamen wir zum Schluss, dass eine Bestellung bei einem chinesischen Hersteller die günstigste Variante ist. Der Hersteller war um den Faktor 10 billiger. Wir haben uns für die Webseite \textit{https://www.seeedstudio.com} entschieden. 
\begin{figure}
	\centering
	\includegraphics[width=0.6\textwidth]{bilder/gerberViewer}
	\caption{Gerber-Viewer – Seeedstudio.com}
	\label{img:GerberViewer}
\end{figure}

Zunächst musste ein das erstellte Layout in das Gerber-Format exportiert werden. Das Gerber Format enthält alle wichtigen Informationen, die zur Produktion der Platine benötigt werden. Es handelt sich bei diesem Format um eine Art Quasi-Standard. Die Gerber Dateien müssen anschließend gezippt werden und können auf die Webseite hochgeladen werden. Die Webeseite bietet ein Gerber-Viewer um die Leiterplatte nach dem Hochladen noch einmal auf die Korrektheit Zur Überprüfen (siehe Bild \ref{img:GerberViewer}). Dieser Schritt war enorm hilfreich, da so festgestellt werden konnte, dass die Beschriftungen nicht richtig skaliert wurden und Sonderzeichen nicht dargestellt wurden. Eine Abhilfe konnte durch eine Umwandlung der Schriften in eine Vektorgrafik erlangt werden. 

Nachdem die Dateien im Gerber-Format hochgeladen wurden mussten noch einige Angaben zur Platine selbst gemacht werden. Diese Angaben waren für beide Platinen die gleichen mit einer Ausnahme:
\begin{itemize}
	\item Material: FR-4 $\rightarrow$ Verbundwerkstoff aus Epozidharz und Glasfasergewebe
	\item Anzahl Schichten: 2
	\item Größe der Platine: 70 mm x 69 mm
	\item Anzahl an Platinen: 10 Stück
	\item Dicke der Platine: 1,6 mm
	\item Farbe der Platine: Weiß (normaler Sensorknoten), Blau (energiesparender Sensorknoten) 
	\item Oberflächenbehandlung: HASL (Hot Air Solder Leveling mit Zinn/Blei)
	\item Mindestabstand zwischen zwei Leiterbahnen: 0,4 mm
	\item Gewicht der Kupferbahnen: $1 oz/ft^2$  entspricht $300 g/m^2$
	\item Mindestgröße von Bohrlöchern: 0,3 mm
\end{itemize}
Nachdem die Eigenschaften alle festgelegt wurden, wurde der Bestellvorgang fortgesetzt. Im letzten Schritt musste noch die Versandmethode gewählt werden. Wir entschieden uns für die günstigste, die jedoch auch die längste Versanddauer hatte. Der Endbetrag errechnete sich dann aus 10,90 Euro Versandkosten und zweimal 4,51 Euro für die Platinen.

Die Platinen wurden am 14. November 2016 bestellt. Am 22. November 2016 waren die Platinen fertiggestellt und wurden verschickt. Den Versand wurde von Singapore Post und DHL übernommen. Die Ware wurde per Luftfracht verschickt und wurde am 21. Dezember 2016 geliefert. Es gab bei der Zustellung jedoch auch Probleme, da bei dem Bestellvorgang die falsche PLZ angegeben wurde und sich deshalb das Paket um eine Woche verzögerte.
\begin{figure}
	\centering
	\includegraphics[width=0.6\textwidth]{bilder/platineArduinoNano}
	\caption{Platine für normaler Sensorknoten}
	\label{img:PlatineArduinoNano}
\end{figure}
\begin{figure}
	\centering
	\includegraphics[width=0.6\textwidth]{bilder/platineArduinoProMini}
	\caption{Platine für energiesparender Sensorknoten}
	\label{img:PlatineArduinoProMini}
\end{figure}

\paragraph{Bestückung de Platinen} Die Platinen entsprachen vollständig dem erwarteten Ergebnisses (siehe Bild \ref{img:PlatineArduinoProMini} und \ref{img:PlatineArduinoNano}). Auf die Platinen konnte extrem einfach und schnell die Steckverbindungen, der Schalter, die Widerstände und den DHT22 Sensor aufgebracht werden. Die Produktionszeit einer Platine reduzierte sich so von 3 Stunden auf 10 Minuten. Zusammenfassend kann eine Empfehlung für \textit{Seeedstudio} ausgesprochen werden. Falls das Produkt jedoch kommerziell genutzt werden sollte, darf aufgrund von neuer EU-Richtlinien nur noch bleifreies Lötzinn verwendet werden. 
