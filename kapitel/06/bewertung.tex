\section{Bewertung des Sensorknotens}
Dieses Unterkapitel bietet eine Bewertung über das Ergebnis der Studienarbeit.
\paragraph{}
Die Studienarbeit kann als Erfolg eingestuft werden, da alle geforderten Anforderungen vollständig erfüllt worden sind. Besonders hervorzuheben sind die selbst designeten und entwickelten Sensorknoten. Hier konnte der komplette Entwicklungs- und Produktionsprozess durchlaufen werden. Beginnend mit der Planung und Umsetzung eines Prototypen, bis zur Fertigstellung eines kompletten Sensorknoten der \textit{Plug-and-Play} verwendet werden kann.

Das Endprodukt der Studienarbeit ist ein Sensorknoten auf Basis von Arduinos, ein Gateway auf Basis eines Raspberry Pi’s, einer Datenhaltungsschicht in Form einer MySQL Datenbank und einer Datenvisualisierung in Form einer Webseite.

Die Vorteile des Sensorknoten ist die Erweiterbarkeit und einfache Befestigung von Sensoren auf dem Sensorknoten. Die energiesparende Sensorknoten können in verschiedenen Szenarien eingesetzt werden, da sie vollständig mit Batterien betrieben werden können. 

Der Raspberry Pi ist mit dem gleichen Funkmodul ausgestattet, dies führt zu einer sehr einfachen Infrastruktur. Sollte Sensoren oder Funkmodule aufgrund von verschiedener Umstände nicht mehr Funktionstüchtig sein können sie einfach ausgetauscht werden.

Die Webseite bietet noch am meisten Erweiterungsmöglichkeiten, da zum Beispiel das hinzufügen neuer Sensorknoten fehlt.

Zusammenfassend lässt sich sagen, dass das Ergebnis das Anforderungsprofil der Studienarbeit übertroffen hat.





